\begin{questions}
\question{
All depends only on one ratio
}
\begin{solution}
Basically that is going to work because is like dividing the original equation by t and solving the new one
\begin{equation}
  \frac{H}{t} = - \sum_\sigma [ c_{1\sigma}^\dagger c_{2\sigma} + c_{2\sigma}^\dagger c_{1\sigma}] + r n_{1\uparrow}n_{1\downarrow} +  r n_{2\uparrow}n_{2\downarrow}.
\end{equation}
So at first glance it seems that the only parameter we have is actually $r$, this is why.

Does this means we solve the same problem? Not exactly. We can see this by looking at the changes it introduces to the energies
\begin{eqnarray}
  \epsilon_1' = \frac{\epsilon_1}{t} = 0,\\
  \epsilon_2' = \frac{\epsilon_2}{t} = r, \\
  \epsilon_i' = \frac{\epsilon_i}{t} = \frac{r}{2} \pm \frac{1}{2}\sqrt{r^2 + 16}, \quad i=3,4.
\end{eqnarray}

If r is the same for different systems the spectrum willl have the same form basically, because we didn't change the topological properties of it. But the non-rescaled energies will change, since we are going to multiply by t in order to get them. So in conclusion, the shape does not change but the value of the energies does.
\end{solution}
\end{questions}

% \includegraphics[width=75mm]{}
%
%
%  \captionof{figure}{}\label{}\vspace{0.5cm}
