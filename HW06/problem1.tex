\begin{questions}
\question{Time evolution of fermionic annihilator
}
\begin{solution}
  Our hamiltonian for this problem is
  \begin{equation}
    H_0 = \sum_{k\sigma} \varepsilon_k c_{k\sigma}^\dagger c_{k\sigma}.
  \end{equation}
  We can prove trivially by using Taylor series for $t$ that
  \begin{equation}
    [e^{\pm iH_0 t},H_0] = 0.
  \end{equation}
  If we take the following relation
  \begin{equation}
    [ABC,D] = AB[C,D] + A[B,D]C + [A,D]BC,
  \end{equation}
  we can easily calculate the next commutator
  \begin{equation}
    \begin{aligned}[b]
      [e^{iH_0 t} c_{k\sigma} e^{- iH_0 t}, H_0] &= e^{iH_0 t} c_{k\sigma}\cancelto{0}{[ e^{- iH_0 t}, H_0]}
      &+ e^{iH_0 t}[ c_{k\sigma} , H_0]e^{- iH_0 t} + \cancelto{0}{[e^{iH_0 t} , H_0]}c_{k\sigma} e^{- iH_0 t},\\
      & = e^{iH_0 t}[ c_{k\sigma} , H_0]e^{- iH_0 t}.
      \label{aux1}
    \end{aligned}
  \end{equation}
  Finally we can use the relation
  \begin{equation}
    [A,BC] = B[A,C] + [A,B]C,
  \end{equation}
  to calculate the following commutator
  \begin{equation}
    \begin{aligned}[b]
      [c_{k\sigma},c_{j\gamma}^\dagger c_{j\gamma}] &=  c_{j\gamma}^\dagger [c_{k\sigma},c_{j\gamma}]+[c_{k\sigma},c_{j\gamma}^\dagger ]c_{j\gamma}, \\
      &= c_{j\gamma}^\dagger (c_{k\sigma}c_{j\gamma} - c_{j\gamma}c_{k\sigma}) + (c_{k\sigma}c_{j\gamma}^\dagger - c_{j\gamma}^\dagger c_{k\sigma})c_{j\gamma},\\
      &= c_{j\gamma}^\dagger c_{k\sigma}c_{j\gamma} + c_{k\sigma}c_{j\gamma}^\dagger c_{j\gamma} - (c_{j\gamma}^\dagger c_{j\gamma}c_{k\sigma}  + c_{j\gamma}^\dagger c_{k\sigma}c_{j\gamma}),\\
      &= \cancelto{\delta_{jk}\delta_{\sigma \gamma}}{\{c_{j\gamma}^\dagger, c_{k\sigma} \}} c_{j\gamma} - c_{j\gamma}^\dagger \cancelto{0}{\{c_{j\gamma}c_{k\sigma}\}},\\
      &= \delta_{jk}\delta_{\sigma \gamma} c_{j\gamma}.
      \label{aux2}
    \end{aligned}
  \end{equation}
  Where in the last step we used the anti-commutation rules of the fermionic creation and annihilation operators.\\

  Now we can finally start. Let's consider the time evolution equation in the Heisenberg picture
  \begin{equation}
    i\frac{d}{dt} c_{k\sigma}(t) = [c_{k\sigma}(t), H_0],
  \end{equation}
  taking
  \begin{equation}
    c_{k\sigma}(t) = e^{iH_0 t} c_{k\sigma} e^{- iH_0 t},
  \end{equation}
  the time evolution equation takes the form
  \begin{equation}
    i\frac{d}{dt} c_{k\sigma}(t) = [e^{iH_0 t} c_{k\sigma} e^{- iH_0 t}, H_0].
  \end{equation}
  Luckily we can use eq. \ref{aux1} to move further and obtain
  \begin{equation}
    i\frac{d}{dt} c_{k\sigma}(t) =  e^{iH_0 t}[ c_{k\sigma} , H_0]e^{- iH_0 t}.
  \end{equation}
  Now is time to plug the hamiltoninan into this commutator
  \begin{equation}
    i\frac{d}{dt} c_{k\sigma}(t) =  e^{iH_0 t}[ c_{k\sigma} , \sum_{k\sigma} \varepsilon_k c_{k\sigma}^\dagger c_{k\sigma}]e^{- iH_0 t},
  \end{equation}
  since the commutator is linear and $\varepsilon_k\in\mathbb{R}$ we can transform the equation to this one
  \begin{equation}
    i\frac{d}{dt} c_{k\sigma}(t) =  e^{iH_0 t}\sum_{k\sigma} \varepsilon_k[ c_{k\sigma} ,  c_{k\sigma}^\dagger c_{k\sigma}]e^{- iH_0 t},
  \end{equation}
  and at this point we are lucky again, because now we can use eq. \ref{aux2} to continue our calculations
  \begin{equation}
    \begin{aligned}[b]
      i\frac{d}{dt} c_{k\sigma}(t) &=  e^{iH_0 t}\sum_{k\sigma} \varepsilon_k \delta_{jk}\delta_{\sigma \gamma} c_{j\gamma}e^{- iH_0 t},\\
      &=  e^{iH_0 t} \varepsilon_k  c_{k\sigma}e^{- iH_0 t},\\
      &=  \varepsilon_k e^{iH_0 t}  c_{k\sigma}e^{- iH_0 t}, \\
      &=  \varepsilon_k c_{k\sigma}(t),
    \end{aligned}
  \end{equation}
  therefore
  \begin{equation}
    \frac{d}{dt} c_{k\sigma}(t) = -i \varepsilon_k c_{k\sigma}(t),
    \label{dif}
  \end{equation}
  at this point the equation looks very familiar, and \textit{if it looks like a duck, swims like a duck, and quacks like a duck, then it probably is a duck}. To make sure we suppose it is
  \begin{equation}
    c_{k\sigma}(t) = e^{\alpha t} c_{k\sigma}.
  \end{equation}
  If we differentiate this operator with respect to $t$, we will have
  \begin{equation}
    \frac{d}{dt} c_{k\sigma}(t) = \frac{d}{dt}e^{\alpha t} c_{k\sigma} = \alpha e^{\alpha t} c_{k\sigma} =\alpha c_{k\sigma}(t).
    \label{dif2}
  \end{equation}
  Comparing eqs. \ref{dif} and \ref{dif2} we finally get that
  \begin{equation}
    \alpha = -i \varepsilon_k,
  \end{equation}
  so finally, our solution will be
  \begin{equation}
    c_{k\sigma}(t) = e^{-i \varepsilon_k t} c_{k\sigma}. \quad _\blacksquare
  \end{equation}
  There are two ways of getting the creation operator, the hard one would be to repeat similar calculations with $c_{k\sigma}^\dagger$, which we will not perform here. The easy one is just to take
  \begin{equation}
    c_{k\sigma}^\dagger(t) = (c_{k\sigma}(t))^\dagger = (e^{-i \varepsilon_k t} c_{k\sigma})^\dagger = e^{i \varepsilon_k t} c_{k\sigma}^\dagger.
  \end{equation}
\end{solution}
\end{questions}


% \includegraphics[width=75mm]{}
%
%
%  \captionof{figure}{}\label{}\vspace{0.5cm}
